%                            COPYRIGHT
%
%  PCB, interactive printed circuit board design
%  Copyright (C) 1994, 1995 Thomas Nau
%  Copyright (C) 1998, 1999, 2000, 2001 harry eaton
%
%  This program is free software; you can redistribute it and/or modify
%  it under the terms of the GNU General Public License as published by
%  the Free Software Foundation; either version 2 of the License, or
%  (at your option) any later version.
%
%  This program is distributed in the hope that it will be useful,
%  but WITHOUT ANY WARRANTY; without even the implied warranty of
%  MERCHANTABILITY or FITNESS FOR A PARTICULAR PURPOSE.  See the
%  GNU General Public License for more details.
%
%  You should have received a copy of the GNU General Public License
%  along with this program; if not, write to the Free Software
%  Foundation, Inc., 675 Mass Ave, Cambridge, MA 02139, USA.
%
%  Contact addresses for paper mail and Email:
%  Thomas Nau, Schlehenweg 15, 88471 Baustetten, Germany
%  Thomas.Nau@rz.uni-ulm.de


\documentstyle[11pt]{article}

%------------------------------------------------------------------
% redefine page size to be A4 landscape, no margins
%
\special{papersize=297mm,210mm}

\textwidth 27.16cm   \hsize\textwidth
\textheight 18.46cm  \vsize\textheight
\voffset -0.2in
\hoffset -0.5in

\topmargin 0cm
\headheight 0cm
\headsep 0cm
\topskip 0cm
\footskip 0cm
\footheight 1cm
\oddsidemargin 0cm
\evensidemargin 0cm
\marginparwidth 0cm

%------------------------------------------------------------------
% size of a single box
%
\def\BoxWidth{8.7cm}
\def\BoxHeight{15cm}
\def\BoxRaise{-\BoxHeight}

%------------------------------------------------------------------
% some new commands to define the modifier keys
%
\newcommand{\Shift}{{\it [S]}}
\newcommand{\Ctrl}{{\it [C]}}
\newcommand{\Mod}{{\it [M]}}
\newcommand{\Btn}{{\it Btn}}
\newcommand{\Fun}{{\it F}}

\thispagestyle{empty}

\begin{document}

\centerline{\Huge\bf PCB-1.99j-mag1 command reference
{\scriptsize \copyright 1998, 1999, 2000, 2001 harry eaton}}
\bigskip
\small

\hfill
\begin{minipage}[t]{\BoxWidth}
\renewcommand{\thefootnote}{\arabic{footnote}}
\fbox{\rule[\BoxRaise]{0cm}{\BoxHeight} \begin{tabular}[t]{ll}
\multicolumn{2}{c}{\bf misc operations} \\
backspace & remove object \\
\Shift\Ctrl\Btn1 & remove object \\
escape & pan (Gumby) mode \\
\Btn1 & create object or part \\
u & undo operation \\
\Shift r & redo operation \\
\Shift\Ctrl u & clear undo-list \\
tab & switch viewing side \\
cursor key & move crosshair 1 grid\\
\Shift cursor key & move crosshair 10 grid\\
 & \\
\multicolumn{2}{c}{\bf connections} \\
\Shift f & reset found connections \\
f & find connections \\
\Shift backspace & remove connections \\
 & \\
\multicolumn{2}{c}{\bf user (:) commands} \\
:DRC() & check layout for rule violations \\
:l [file] & load data file \\
:le [file] & load element to buffer \\
:m [file] & load layout to buffer \\
:q & quit application \\
:rn [file] & load netlist \\
:s [file] & save data as file \\
 & \\
\multicolumn{2}{c}{\bf display} \\
c & center display \\
g & increase grid spacing \\
\Shift g & decrease grid spacing \\
\Ctrl m & mark location \\
r & clear and redraw output \\
z & zoom in \\
\Shift z & zoom out \\
v & zoom extents \\
\end{tabular}}
\end{minipage}
%
%
%
\hfill
\begin{minipage}[t]{\BoxWidth}
\fbox{\rule[\BoxRaise]{0cm}{\BoxHeight} \begin{tabular}[t]{ll}
\multicolumn{2}{c}{\bf selection} \\
\Btn2 & select/deselect object \\
\Shift\Btn2 & add object to selection \\
drag \Btn2 & select only objects in box \\
drag \Shift\Btn2 & add box to selection \\
\Shift m & move selected to current layer \\
 & \\
\multicolumn{2}{c}{\bf copy and move} \\
drag \Btn2 & move object or selection\\
drag \Mod\Btn2 & copy object \\
drag \Shift\Mod\Btn2 & override rubberband \& move \\
m & move to current layer \\
 & \\
\multicolumn{2}{c}{\bf pastebuffer} \\
\Fun3 & enter pastebuffer-mode \\
\Shift \Fun3 & rotate 90 degree cc \\
\Btn1 & copy to layout \\
\Shift 1$\cdots$5 & select buffer \# 1$\cdots$5 \\
x & copy selected objects to buffer \\
\Shift x & cut selected objects to buffer \\
 & \\
\multicolumn{2}{c}{\bf sizing} \\
s & increase size of TLAPV \footnote{TLAPV: text, line, arc, pin or via}\\
\Shift s & decrease size of TLAPV\\
\Mod s & increase drill size of PV \\
\Shift\Mod s & decrease drill size of PV \\
k & increase clearance of LAPV\\
\Shift\ k & decrease clearance of LAPV\\
& \\
\multicolumn{2}{c}{\bf element} \\
d & display pinout \\
\Shift d & open pinout window \\
h & hide/show element name \\
n & change element name \\
 & \\
\multicolumn{2}{c}{\bf pin/pad} \\
n & change name \\
q & toggle square flag \\
\end{tabular}}
\end{minipage}
%
%
%
\hfill
\begin{minipage}[t]{\BoxWidth}
\fbox{\rule[\BoxRaise]{0cm}{\BoxHeight} \begin{tabular}[t]{ll}
\multicolumn{2}{c}{\bf via} \\
\Fun1 & enter via-mode \\
\Ctrl v & increase initial size \\
\Shift v & decrease initial size \\
\Mod v & inc. initial drilling hole \\
\Shift\Mod v & dec. initial drilling hole \\
\Ctrl h & convert via to mounting hole \\
 & \\
\multicolumn{2}{c}{\bf lines/arcs} \\
\Fun2 & enter line mode \\
\Fun8 & enter arc mode \\
l & increase initial line size \\
\Shift l & decrease initial line size \\
period & toggle 45 degree enforcement \\
/ & cycle multiline mode \\
\Shift & override multiline mode \\
 & \\
\multicolumn{2}{c}{\bf polygon} \\
\Fun4 & enter rectangle-mode \\
\Fun6 & enter polygon-mode \\
\Shift p & close path \\
insert & enter insert point mode \\
 & \\
\multicolumn{2}{c}{\bf text} \\
\Fun5 & enter text-mode \\
n & edit string \\
t & increase initial text size \\
\Shift t & decrease initial text size \\
 & \\
\multicolumn{2}{c}{\bf rats nest} \\
w & add all rats \\
\Shift w & add rats to selected pins/pads \\
e & delete all rats \\
\Shift e & delete selected rats \\
o & optimize all rats \\
\Shift o & optimize selected rats \\
\end{tabular}}
\end{minipage}
\hfill

\medskip
Obviously \Shift, \Ctrl, \Mod, \Fun \space and \Btn \space mean the shift,
control, modifier1 (BTNMOD for buttons), function key and mouse button.

\end{document}
